\documentclass[../main.tex]{subfiles}
\begin{document}

\section[Problem 2]{Let $T_{1}:M_{22}->P_{1}$ and $T_{2}:P_{1}->\mathbb{R}^3$ be linear transformations}
given by
\begin{align*}
  T_{1}\left(\begin{bmatrix}
      a & b \\
      c & d
    \end{bmatrix}\right) & =( a+b) +( c+d) x \\
  T_{2}( a+bx)                                & =( a,b,a)
\end{align*}
% %%%%%%%
% Problem 
% 2.a
% %%%%%%%
\subsection{Find the formula for $T_{2} \circ T_{1}$ }
\boxed{Solution:( a+b,\ c+d,\ a+b)}
\begin{proof}

  \begin{equation*}
    \begin{aligned}
      ( T_{2} \circ T_{1}) & =T_{2}\left( T_{1}\left(\begin{bmatrix}
          a & b \\
          c & d
        \end{bmatrix}\right)\right) =T_{2}(( a+b) +( c+d) x)
    \end{aligned}
  \end{equation*}
  let $\displaystyle u=a+b$ and let $\displaystyle v=( c+d)$
  \begin{align*}
    \therefore T_{2}( u+vx) & =( u,\ v,\ u) =( a+b,\ c+d,\ a+b)
  \end{align*}
\end{proof}
% %%%%%%%
% Problem 
% 2.b
% %%%%%%%
\subsection{Show that $T_{2} \circ T_{1}$ is not one-to-one by finding distinct $2 \times 2$ matricies $A$ and $B$ such that $T_{2} \circ T_{1}(A)=T_{2} \circ T_{1}(B)$}
\boxed{Solution:\displaystyle A=\begin{bmatrix}
    -1 & 1 \\
    -2 & 2
  \end{bmatrix}
  \ \displaystyle B=\begin{bmatrix}
    -2 & 2 \\
    -1 & 1
  \end{bmatrix}}
\begin{proof}
  We will find a basis for $\displaystyle ker( T)$, and use the basis vectors to construct our matricies $\displaystyle A$ and $\displaystyle B$ such that $\displaystyle ( T_{2} \circ T_{1})( A) =( T_{2} \circ T_{1})( B)$
  \begin{equation*}
    \begin{aligned}
      ( T_{2} \circ T_{1}) & =\begin{bmatrix}
        1 & 1 & 0 & 0 \\
        0 & 0 & 1 & 1 \\
        1 & 1 & 0 & 0
      \end{bmatrix} ;\ rref(( T_{2} \circ T_{1})) =\begin{bmatrix}
        1 & 1 & 0 & 0 \\
        0 & 0 & 1 & 1 \\
        0 & 0 & 0 & 0
      \end{bmatrix}
    \end{aligned}
  \end{equation*}
  So a basis for $\displaystyle ker( T)$ is $\displaystyle \left\{\begin{bmatrix}
      -1 & 1 & 0 & 0
    \end{bmatrix}^{T} ,\ \begin{bmatrix}
      0 & 0 & -1 & 1
    \end{bmatrix}^{T}\right\}$. Thus any matricies with bases $\displaystyle \left\{b\begin{bmatrix}
      -1 & 1 \\
      0  & 0
    \end{bmatrix} ,\ d\begin{bmatrix}
      0  & 0 \\
      -1 & 1
    \end{bmatrix}\right\}$ satisfy our problem. So let $\displaystyle A=\begin{bmatrix}
      -1 & 1 \\
      -2 & 2
    \end{bmatrix}$ and $\displaystyle B=\begin{bmatrix}
      -2 & 2 \\
      -1 & 1
    \end{bmatrix}$,
  \begin{equation*}
    \begin{aligned}
      ( T_{2} \circ T_{1})( A) & =( -1+1,\ -2+2,\ -1+1) \\
                               & =( 0,\ 0,\ 0)          \\
      ( T_{2} \circ T_{1})( B) & =( -2+2,\ -1+1,\ -2+2) \\
                               & =( 0,\ 0,\ 0)
    \end{aligned}
  \end{equation*}
  $\displaystyle \therefore ( T_{2} \circ T_{1})( A) =( T_{2} \circ T_{1})( B)$
\end{proof}
% %%%%%%%
% Problem 
% 2.c
% %%%%%%%
\subsection{Show that $T_{2} \circ T_{1}$ is not onto by finding a vector $(a,b,c) \in \mathbb{R}^3$ that is not in the range $T_{2} \circ T_{1}$}
\boxed{Solution: (1, 2, 3)^{T}}
\begin{proof}
  Since $\displaystyle ( T_{2} \circ T_{1}) =( a+b,\ c+d,\ a+b)$ any vector $\displaystyle \vec{v} =( a,\ b,\ c)^{T} :\ a\neq c$ is not in the range of $\displaystyle ( T_{2} \circ T_{1})$, to show this take the vector $\displaystyle \vec{v} =( 1,\ 2,\ 3)^{T}$ and construct the augmented matrix
  \begin{equation*}
    A=\begin{bmatrix}
      1 & 1 & 0 & 0 & \bigm| & 1 \\
      0 & 0 & 1 & 1 & \bigm| & 2 \\
      1 & 1 & 0 & 0 & \bigm| & 3
    \end{bmatrix} ;\ rref( A) =\begin{bmatrix}
      1 & 1 & 0 & 0 & \bigm| & 0 \\
      0 & 0 & 1 & 1 & \bigm| & 0 \\
      0 & 0 & 0 & 0 & \bigm| & 1
    \end{bmatrix}
  \end{equation*}
  Clearly this has no solutions because $\displaystyle 0\neq 1$, therefore $\displaystyle ( T_{2} \circ T_{1})$ is not onto.
\end{proof}
\end{document}