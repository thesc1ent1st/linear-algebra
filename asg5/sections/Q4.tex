\documentclass[../main.tex]{subfiles}
\begin{document}

\section[Problem 4]{Consider the linear transformation $D: \mathcal{P}_3 \to \mathcal{P}_2$ given by $D(p(x)) = p'(x)$ (i.e. differentiation). Consider also the linear transformation $S: \mathcal{P}_2 \to \mathcal{P}_3$ given by \DS{S(p(x)) = \int_0^x p(t)dt}}
% %%%%%%%
% Problem 
% 4.a
% %%%%%%%
\subsection{For an arbitrary $p(x) \in \mathcal{P}_2$, what is $(D \circ S)(p(x))$? What is another name for the transformation $D \circ S$?}
\begin{equation*}
        \boxed{ Solutions:}
\end{equation*}
\begin{proof}
\end{proof}
\pagebreak
% %%%%%%%
% Problem 
% 4.b
% %%%%%%%
\subsection{For an arbitrary $p(x) \in \mathcal{P}_3$, what is $(S \circ D)(p(x))$? What can you say about the transformation $S \circ D$? Compare to part (a).}
\begin{equation*}
        \boxed{ Solutions:}
\end{equation*}
\begin{proof}
\end{proof}
% %%%%%%%
% Problem 
% 4.c
% %%%%%%%
\subsection{Are $S$ and $D$ inverses? Explain.}
\begin{equation*}
        \boxed{ Solutions:}
\end{equation*}
\begin{proof}
\end{proof}
% %%%%%%%
% Problem 
% 4.d
% %%%%%%%
\subsection{Find a basis for Range($S$).}
\begin{equation*}
        \boxed{ Solutions:}
\end{equation*}
\begin{proof}
\end{proof}
% %%%%%%%
% Problem 
% 4.e
% %%%%%%%
\subsection{Find a basis for ker($D$), and for ${\rm ker}(D)^{\perp}$ (with respect to the standard inner product - see example 7 from section 6.1). Compare to part (d).}
\begin{equation*}
        \boxed{ Solutions:}
\end{equation*}
\begin{proof}
\end{proof}
% %%%%%%%
% Problem 
% 4.f
% %%%%%%%
\subsection{Show that $S$ is one-to-one. (and therefore $S: \mathcal{P}_2 \to {\rm Range}(S)$ is an isomorphism).}
\begin{equation*}
        \boxed{ Solutions:}
\end{equation*}
\begin{proof}
\end{proof}
% %%%%%%%
% Problem 
% 4.g
% %%%%%%%
\subsection{Verify that $D: {\rm Range}(S) \to \mathcal{P}_2$ and $S: \mathcal{P}_2 \to {\rm Range}(S)$ are inverses.}
\begin{equation*}
        \boxed{ Solutions:}
\end{equation*}
\begin{proof}
\end{proof}
% %%%%%%%
% Problem 
% 4.h
% %%%%%%%
\subsection{If we use the standard inner products on $\mathcal{P}_3$ and $\mathcal{P}_2$, is $D$ an inner product space isomorphism? Justify your answer.}
\begin{equation*}
        \boxed{ Solutions:}
\end{equation*}
\begin{proof}
\end{proof}
\end{document}