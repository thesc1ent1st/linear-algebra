\documentclass[../main.tex]{subfiles}
\begin{document}

\section[Problem 4]{Consider the linear transformation $D: \mathcal{P}_3 \to \mathcal{P}_2$ given by $D(p(x)) = p'(x)$ (i.e. differentiation). Consider also the linear transformation $S: \mathcal{P}_2 \to \mathcal{P}_3$ given by \DS{S(p(x)) = \int_0^x p(t)dt}}
% %%%%%%%
% Problem 
% 4.a
% %%%%%%%
\subsection{For an arbitrary $p(x) \in \mathcal{P}_2$, what is $(D \circ S)(p(x))$? What is another name for the transformation $D \circ S$?}
\boxed{ Solution: a_{0} +a_{1} x+a_{2} x^{2}}
\begin{proof}
        let $\displaystyle \begin{aligned}
                        p( x) & =a_{0} +a_{1} x+a_{2} x^{2}
                \end{aligned}$
        \begin{align*}
                ( D\circ S)( p( x)) & =D\left( S\left( a_{0} +a_{1} x+a_{2} x^{2}\right)\right) =D\left(\int _{0}^{x}\left( a_{0} +a_{1} t+a_{2} t^{2}\right) dt\right)                                          \\
                                    & =D\left(\left[ a_{0} t+\frac{1}{2} a_{1} t^{2} +\frac{1}{3} a_{2} t^{3}\right]_{t=0}^{t=x}\right) =D\left( a_{0} x+\frac{1}{2} a_{1} x^{2} +\frac{1}{3} a_{2} x^{3}\right) \\
                                    & =\frac{d}{dx}\left( a_{0} x+\frac{1}{2} a_{1} x^{2} +\frac{1}{3} a_{2} x^{3}\right) =a_{0} +a_{1} x+a_{2} x^{2}
        \end{align*}
        Another name for the transformation $\displaystyle ( D\circ S)( p( x))$ is the fundamental theorem of calculus.
\end{proof}
% %%%%%%%
% Problem 
% 4.b
% %%%%%%%
\subsection{For an arbitrary $p(x) \in \mathcal{P}_3$, what is $(S \circ D)(p(x))$? What can you say about the transformation $S \circ D$? Compare to part (a).}
\boxed{Solution: a_{1} x+a_{2} x^{2} +a_{3} x^{3}}
\begin{proof}
        et $\displaystyle p( x) =a_{0} +a_{1} x+a_{2} x^{2} +a_{3} x^{3}$
        \begin{align*}
                ( S\circ D)( p( x)) & =S\left( D\left( a_{0} +a_{1} x+a_{2} x^{2} +a_{3} x^{3}\right)\right) =S\left(\frac{d}{dx}\left( a_{0} +a_{1} x+a_{2} x^{2} +a_{3} x^{3}\right)\right)  \\
                                    & =S\left( a_{1} +2a_{2} x+3a_{3} x^{2}\right) =S\left( a_{1} +2a_{2} x+3a_{3} x^{2}\right)                                                                \\
                                    & =\int _{0}^{x}\left( a_{1} +2a_{2} t+3a_{3} t^{2}\right) dt=\left[ a_{1} t+a_{2} t^{2} +a_{3} t^{3}\right]_{t=0}^{t=x} =a_{1} x+a_{2} x^{2} +a_{3} x^{3} \\
                                    & =a_{1} x+a_{2} x^{2} +a_{3} x^{3}
        \end{align*}
        The two operations are inverses of each other apart from a constant value.
\end{proof}
% %%%%%%%
% Problem 
% 4.c
% %%%%%%%
\subsection{Are $S$ and $D$ inverses? Explain.}
\begin{proof}
        D is not one-to-one, so it does not have an inverse.
\end{proof}
% %%%%%%%
% Problem 
% 4.d
% %%%%%%%
\subsection{Find a basis for Range($S$).}
\boxed{Solution: \displaystyle B=\left\{x,\ \frac{1}{2} x^{2} ,\ \frac{1}{3} x^{3}\right\}}
\begin{proof}
        \begin{align*}
                S( p( x)) & =a_{1} x+\frac{1}{2} a_{2} x^{2} +\frac{1}{3} a_{3} x^{3}
        \end{align*}
        Thus, by inspection
        \begin{align*}
                Range( S) & =span\left\{x,\ \frac{1}{2} x^{2} ,\ \frac{1}{3} x^{3}\right\}
        \end{align*}
\end{proof}
% %%%%%%%
% Problem 
% 4.e
% %%%%%%%
\subsection{Find a basis for ker($D$), and for ${\rm ker}(D)^{\perp}$ (with respect to the standard inner product - see example 7 from section 6.1). Compare to part (d).}
\boxed{Solution:ker( D)= span\{1\}\ \land\ ker( D)^{\perp } =span\left\{x,\ 2x^{2} ,\ 3x^{3}\right\}}
\begin{proof}
        Since $\displaystyle ker( D)$ is all vectors,
        \begin{align*}
                a_{1} +2a_{2} x+3a_{3} x^{2} & =\vec{0} \\
                a_{1} =a_{2} =a_{3}          & =0
        \end{align*}
        by inspection
        \begin{align*}
                ker( D)          & =span\{1\}                               \\
                ker( D)^{\perp } & =span\left\{x,\ 2x^{2} ,\ 3x^{3}\right\}
        \end{align*}
        Finaly, to compare the basis for $\displaystyle ker( D)^{\perp }$ to a basis for $\displaystyle Range( S)$
        \begin{equation*}
                let\ \vec{v} =a_{1} x+2a_{2} x+3a_{3} x^{3} ,\ and\ let\ \vec{u} =b_{1} x+\frac{1}{2} b_{2} x+\frac{1}{3} b_{3} x^{3}
        \end{equation*}
        and take the standard inner product.
        \begin{align*}
                \langle \vec{v} ,\vec{u} \rangle & =a_{1} b_{1} +a_{2} b_{2} +a_{3} b_{3}                               \\
                                                 & =\begin{bmatrix}
                        a_{1} & a_{2} & a_{3}
                \end{bmatrix}^{T} \cdot \begin{bmatrix}
                        b_{1} & b_{2} & b_{3}
                \end{bmatrix}^{T}
        \end{align*}
        Therefore, these vectors share the same basis.
\end{proof}
% %%%%%%%
% Problem 
% 4.f
% %%%%%%%
\subsection{Show that $S$ is one-to-one. (and therefore $S: \mathcal{P}_2 \to {\rm Range}(S)$ is an isomorphism).}
\boxed{Solution:ker( S) =span\{0\}}
\begin{proof}
        We know ${\displaystyle S:\mathcal{P}_{2}\rightarrow \mathcal{P}_{3}}$ gives some vector ${\displaystyle \vec{p} =a_{0} x+\frac{1}{2} a_{1} x^{2} +\frac{1}{3} a_{2} x^{3}}$. \\\\
        But, the basis of a kernel for all vectors from $\displaystyle \mathcal{P}_{2}$ concerning a transformation of ${\displaystyle S:\mathcal{P}_{2}\rightarrow \mathcal{P}_{3}}$ has the property
        \begin{equation*}
                \begin{aligned}
                        {\displaystyle a_{0} =a_{1} =a_{2} =\vec{0}} & \rightarrow ker( S) =span\{0\}
                \end{aligned}
        \end{equation*}
        Thus, ${\displaystyle S}$ must be one-to-one.
\end{proof}
% %%%%%%%
% Problem
% 4.g
% %%%%%%%
\subsection{Verify that $D: {\rm Range}(S) \to \mathcal{P}_2$ and $S: \mathcal{P}_2 \to {\rm Range}(S)$ are inverses.}
\begin{proof}
        Consider the basis
        \begin{equation*}
                S=\left\{x,\ \frac{1}{2} x^{2} ,\ \frac{1}{3} x^{3}\right\}
        \end{equation*}
        and let,
        \begin{align*}
                r( x) & =a_{0} x+\frac{1}{2} a_{1} x^{2} +\frac{1}{3} a_{2} x^{3}
        \end{align*}
        be a linear combination of the basis vectors in $\displaystyle S$, so $\displaystyle \vec{r} \in Range( S)$.\\\\
        Then, we obtain $\displaystyle D:Range( S)\rightarrow \mathcal{P}_{2}$ such that,
        \begin{align*}
                D( r( x)) & =\frac{d}{dx}\left[ a_{0} x+\frac{1}{2} a_{1} x^{2} +\frac{1}{3} a_{2} x^{3}\right] \\
                          & =a_{0} +a_{1} x+a_{2} x^{2}
        \end{align*}
        finally, let
        \begin{equation*}
                \vec{p} \in \mathcal{P}_{2} \ :\ p( x) =a_{0} +a_{1} x+a_{2} x^{2}
        \end{equation*}
        Then, we obtain $\displaystyle S:\mathcal{P}_{2}\rightarrow Range( S)$ such that,
        \begin{align*}
                S( p( x)) & =\int _{0}^{x}\left( a_{0} +a_{1} t+a_{2} t^{2}\right) dt=\left[ a_{0} t+\frac{1}{2} a_{1} t^{2} +\frac{1}{3} t^{3}\right]_{t=0}^{t=x} \\
                          & =a_{0} x+\frac{1}{2} a_{1} x^{2} +\frac{1}{3} x^{3}
        \end{align*}
        Clearly,
        \begin{equation*}
                {\displaystyle \int _{0}^{x}\vec{p} \cdot dt=\vec{r}} \ \land \ {\displaystyle \frac{d}{dx} [\vec{r} ]=\vec{p}} .
        \end{equation*}Thus, ${\displaystyle D:Range(S)\rightarrow \mathcal{P}_{2}}$ and ${\displaystyle S:\mathcal{P}_{2}\rightarrow Range(S)}$ are inverses.
\end{proof}
% %%%%%%%
% Problem 
% 4.h
% %%%%%%%
\subsection{If we use the standard inner products on $\mathcal{P}_3$ and $\mathcal{P}_2$, is $D$ an inner product space isomorphism? Justify your answer.}
\boxed{Solutions: no}
\begin{proof}
        $\displaystyle D:\mathcal{P}_{3}\rightarrow \mathcal{P}_{2}$ is an inner product space isomorphism if it is an isomorphism of vector spaces, and
        \begin{equation*}
                \langle t(\vec{p}) ,\ T(\vec{q}) \rangle _{p_{2}} =\langle \vec{p} ,\ \vec{q} \rangle _{p_{3}} \ \forall \ \vec{u} ,\vec{v} \in V
        \end{equation*}
        Since $\displaystyle dim(\mathcal{P}_{3})  >dim(\mathcal{P}_{2})$, we conclude $\displaystyle D$ is not $\displaystyle one-to-one$. Therefore, $\displaystyle \mathcal{P}_{2}$ is not isomorphic to $\displaystyle \mathcal{P}_{3}$. Thus, $\displaystyle D$ can't be an inner product space isomorphism.
\end{proof}
\end{document}