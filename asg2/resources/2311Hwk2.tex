\documentclass[12pt]{article}

\usepackage{latexsym}
\usepackage{amsfonts}
\usepackage{amsmath, amsthm, amssymb}


\setlength{\textheight}{10in}
\setlength{\textwidth}{6.5in}
\setlength{\topmargin}{-1.0in}
\setlength{\parskip}{0.15in}
\setlength{\oddsidemargin}{-0.1in}

\newcommand {\DS} [1] {${\displaystyle #1}$}

\begin{document}
\thispagestyle{empty}

\begin{center}
\textbf{Math 2311  - Assignment 2}\\
Due: Wednesday, Feb. 9th, 2022.
\end{center}


\begin{enumerate}

\item
Show that \DS{S = \{ 1+2x+x^2, 2 + 9x, 3 + 3x + 4x^2 \} } is a basis for $P_2$ and write \\ 
$p = 2 + 17x - 3x^2$ as a linear combination of vectors in $S$. Finally, write $[p]_S$.

\item
Recall the standard basis of  $\mathbb{R}^3$, \DS{ \vec{e_1} = [1 \ 0 \ 0]^T, \ \vec{e_2} = [0 \ 1 \ 0]^T, \ \vec{e_3} = [0 \ 0 \ 1]^T}
\begin{enumerate}
\item Consider the matrix \DS{ A = \left[\begin{array}{ccc}
1 & 0 & 2 \\
1 & 3 & 2 \\
0 & 3 & 2
\end{array}\right]}. Does the set of vectors \DS{S_1 = \{ A\vec{e_1},A\vec{e_2},A\vec{e_3} \}} form a basis for $\mathbb{R}^3$?

%the above and below show different ways of creating a matrix. The below is a bit more efficient, but they are not identical - the above leaves slightly more space between the entries of the matrix and the brackets on either side.

\item Consider the matrix \DS{ B = \begin{bmatrix}
1 & 0 & 2 \\
1 & 3 & 0 \\
0 & 3 & -2
\end{bmatrix}}. Does the set of vectors \DS{S_2 = \{ B\vec{e_1},B\vec{e_2},B\vec{e_3} \}} form a basis for $\mathbb{R}^3$?

\item Make a conjecture of the form ``\DS{S = \{ A\vec{e_1},A\vec{e_2},A\vec{e_3} \}} forms a basis for $\mathbb{R}^3$ if and only if $A$ \emph{(insert appropriate property of A here)}''. \\
Bonus: Prove your conjecture.
 
\end{enumerate}

\item For each of the following subspaces of $M_{33}$ find a basis and state the dimension. You should check for yourself that it is in fact a basis, but this verification does not need to be a part of your submitted answer.
\begin{enumerate}
\item \DS{W_1 = \{A \in M_{33} | A \ \textrm{ is a diagonal matrix} \} }
\item \DS{W_2 = \{A \in M_{33} | A = A^T \} } (the symmetric matrices)
\item \DS{W_1 = \{A \in M_{33} | A = -A^T \} } (the anti-symmetric matrices)
\end{enumerate}

%\item
%In each part, for the given $A$, find the dimension of the vector space consisting of all $\vec{x}$ such that $A\vec{x} = \vec{0}$.
%\begin{enumerate}
%\item \DS{ A = \begin{bmatrix}
%1 & 0 & 2 & -1 \\
%-1 & 4 & 0 & 0
%\end{bmatrix}}
%\item \DS{ A = \begin{bmatrix}
%0 & 0 & 1 & 1 \\
%-1 & 1 & 0 & 0 \\
%1 & 0 & 0 & 1
%\end{bmatrix}}
%\end{enumerate}

\item Find a basis for the subspace of $P_3$ spanned by the following polynomials (vectors):
$$
p_1 = 1 + x + 3x^2 + 4x^3, \ p_2 = 1 + 2x^2 + 3x^3, \ p_3 = x + x^2 + 2x^3, \ p_4 = 1 + x + 3x^2 + 5x^3
$$


\item
Let $\vec{x} = [1 \ 2 \ 3]^T$, $\mathcal{B} = \{[1 \ 0 \ 0]^T, \  [1 \ 1 \ 0]^T, \ [1 \ 1 \ 1]^T \}$, and $\mathcal{C} = \{[1 \ 1 \ 0]^T, \  [0 \ 1 \ 1]^T, \ [1 \ 0 \ 1]^T \}$.
\begin{enumerate}
\item Find $[\vec{x}]_{\mathcal{B}}$ 
\item Find $[\vec{x}]_{\mathcal{C}}$ 
\item Find $P_{\mathcal{C} \leftarrow \mathcal{B}}$ and compute $P_{\mathcal{C} \leftarrow \mathcal{B}}[\vec{x}]_{\mathcal{B}}$ 
\item Find $P_{\mathcal{B} \leftarrow \mathcal{C}}$ and compute $P_{\mathcal{B} \leftarrow \mathcal{C}}[\vec{x}]_{\mathcal{C}}$
\end{enumerate}

\item Let \DS{ A = \begin{bmatrix}
1 & 4 & 5 & 2 \\
2 & 1 & 3 & 0 \\
-1 & 3 & 2 & 2
\end{bmatrix}}
\begin{enumerate}
\item Find a basis for each of null(A), row(A), col(A), and state the dimension of each of these subspaces.
\item Is the vector \DS {\vec{b} = \begin{bmatrix}
4 \\
6 \\
-2
\end{bmatrix}} in the column space of $A$? If so, write $\vec{b}$ as a linear combination of the columns of $A$.
\end{enumerate}
\end{enumerate}
\end{document}
