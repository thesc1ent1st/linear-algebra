\documentclass[../main.tex]{subfiles}
\begin{document}

\section{Let $\vec{x} = [\ 1 \ 2 \ 3\ ]^T$, $\mathcal{B} = \{[\ 1 \ 0 \ 0\ ]^T, \  [\ 1 \ 1 \ 0\ ]^T, \ [\ 1 \ 1 \ 1\ ]^T \}$, and $\mathcal{C} = \{[\ 1 \ 1 \ 0\ ]^T, \  [\ 0 \ 1 \ 1\ ]^T, \ [\ 1 \ 0 \ 1\ ]^T \}$.}
\subsection{Find $[\vec{x}]_{\mathcal{B}}$}
\fbox{Solution: $[\vec{x}]_{\mathcal{B}} = [\ -1 \ -1 \ 3\ ]^T$}
        \begin{proof} By inspection: $\vec{x} = (-1)[\ 1 \ 0 \ 0\ ]^T + (-1)[\ 1 \ 1 \ 0\ ]^T + (3)[\ 1 \ 1 \ 1\ ]^T [\vec{x}]_{\mathcal{B}} =  [\ -1 \ -1 \ 3\ ]^T $
        \end{proof}
        \subsection{Find $[\vec{x}]_{\mathcal{C}}$}
        \fbox{Solution: $[\vec{x}]_{\mathcal{C}} = [\ 0 \ 2 \ 1\ ]^T$}
\begin{proof}
        By inspection: $\vec{x} =(0)[\ 1 \ 1 \ 0\ ]^T + (2)[\ 0 \ 1 \ 1\ ]^T + (1)[\ 1 \ 0 \ 1\ ]^T$
        $\implies [\vec{x}]_{\mathcal{C}} = [\ 0 \ 2 \ 1\ ]^T$
\end{proof}
\subsection{Find $P_{\mathcal{C} \leftarrow \mathcal{B}}$ and compute $P_{\mathcal{C} \leftarrow \mathcal{B}}[\vec{x}]_{\mathcal{B}}$}
\begin{align*}
        \boxed{
        Solutions: P_{\mathcal{C} \leftarrow \mathcal{B}} =
        \begin{bmatrix}
                \ {1}/{2}  & 1 & {1}/{2} \ \\
                \ -{1}/{2} & 0 & {1}/{2} \ \\
                \ {1}/{2}  & 0 & {1}/{2} \
        \end{bmatrix}; [\vec{x}]_\mathcal{C} = P_{\mathcal{C} \leftarrow \mathcal{B}}[\vec{x}]_{\mathcal{B}} = [\ 0 \ 2 \ 1\ ]^T
                }
\end{align*}
\begin{proof}
        \begin{align*}
                \textrm{Partitioned matrix } [\ \mathcal{C}\ |\ \mathcal{B}\ ] =
                 & \begin{bmatrix}
                        \ 1 & 0 & 1 & \bigm| & 1 & 1 & 1 \ \\
                        \ 1 & 1 & 0 & \bigm| & 0 & 1 & 1 \ \\
                        \ 0 & 1 & 1 & \bigm| & 0 & 0 & 1 \
                \end{bmatrix} \\
                \textrm{Transition matrix } [\ I_3 \ |\ {\mathcal{C} \leftarrow \mathcal{B}}\ ] =
                 & \begin{bmatrix}
                        \ 1 & 0 & 0 & \bigm| & {1}/{2}  & 1 & {1}/{2} \ \\
                        \ 0 & 1 & 0 & \bigm| & -{1}/{2} & 0 & {1}/{2} \ \\
                        \ 0 & 0 & 1 & \bigm| & {1}/{2}  & 0 & {1}/{2} \
                \end{bmatrix} \\
                P_{\mathcal{C} \leftarrow \mathcal{B}} =
                 & \begin{bmatrix}
                        \ {1}/{2}  & 1 & {1}/{2} \ \\
                        \ -{1}/{2} & 0 & {1}/{2} \ \\
                        \ {1}/{2}  & 0 & {1}/{2} \
                \end{bmatrix}
        \end{align*}
        \begin{align*}
                [\vec{x}]_\mathcal{C} = P_{\mathcal{C} \leftarrow \mathcal{B}}[\vec{x}]_{\mathcal{B}}
                 & = \begin{bmatrix}
                        \ {1}/{2}  & 1 & {1}/{2} \ \\
                        \ -{1}/{2} & 0 & {1}/{2} \ \\
                        \ {1}/{2}  & 0 & {1}/{2} \
                \end{bmatrix}[\ -1 \ -1 \ 3\ ]^T \\
                 & = [-1]\begin{bmatrix}
                        \ {1}/{2} \ \\
                        \ -{1}/{2}\ \\
                        \ {1}/{2} \
                \end{bmatrix} +
                [-1] \begin{bmatrix}
                        \ 1 \ \\
                        \ 0 \ \\
                        \ 0 \
                \end{bmatrix} +
                [3]\begin{bmatrix}
                        \ {1}/{2} \ \\
                        \ {1}/{2} \ \\
                        \ {1}/{2} \
                \end{bmatrix}                     \\
                 & = [\ 0 \ 2 \ 1\ ]^T
        \end{align*}
\end{proof}
\pagebreak
\subsection{Find $P_{\mathcal{B} \leftarrow \mathcal{C}}$ and compute $P_{\mathcal{B} \leftarrow \mathcal{C}}[\vec{x}]_{\mathcal{C}}$}
\begin{align*}
        \boxed{
        Solutions: P_{\mathcal{B} \leftarrow \mathcal{C}} =
        \begin{bmatrix}
                \ 0 & -1 & 1 \  \\
                \ 1 & 0  & -1 \ \\
                \ 0 & 1  & 1  \
        \end{bmatrix}; [\vec{x}]_\mathcal{B} = P_{\mathcal{B} \leftarrow \mathcal{C}}[\vec{x}]_{\mathcal{C}} = [\ -1 \ -1 \ 3\ ]^T
                }
\end{align*}\
\begin{proof}
        \begin{align*}
                \textrm{Partitioned matrix } [\ \mathcal{B}\ |\ \mathcal{C}\ ] =
                 & \begin{bmatrix}
                        \ 1 & 1 & 1 & \bigm| & 1 & 0 & 1 \ \\
                        \ 0 & 1 & 1 & \bigm| & 1 & 1 & 0 \ \\
                        \ 0 & 0 & 1 & \bigm| & 0 & 1 & 1 \
                \end{bmatrix} \\
                \textrm{Transition matrix } [\ I_3 \ |\ {\mathcal{B} \leftarrow \mathcal{C}}\ ] =
                 & \begin{bmatrix}
                        \ 1 & 0 & 0 & \bigm| & 0 & -1 & 1 \  \\
                        \ 0 & 1 & 0 & \bigm| & 1 & 0  & -1 \ \\
                        \ 0 & 0 & 1 & \bigm| & 0 & 1  & 1 \
                \end{bmatrix} \\
                P_{\mathcal{B} \leftarrow \mathcal{C}} =
                 & \begin{bmatrix}
                        \ 0 & -1 & 1 \  \\
                        \ 1 & 0  & -1 \ \\
                        \ 0 & 1  & 1  \
                \end{bmatrix}
        \end{align*}
        \begin{align*}
                [\vec{x}]_\mathcal{B} = P_{\mathcal{B} \leftarrow \mathcal{C}}[\vec{x}]_{\mathcal{C}}
                 & = \begin{bmatrix}
                        \ 0 & -1 & 1 \  \\
                        \ 1 & 0  & -1 \ \\
                        \ 0 & 1  & 1  \
                \end{bmatrix}[\ 0 \ 2 \ 1\ ]^T \\
                 & = [0]\begin{bmatrix}
                        \ 0 \ \\
                        \ 1 \ \\
                        \ 0 \
                \end{bmatrix} +
                [2] \begin{bmatrix}
                        \ -1 \ \\
                        \ 0  \ \\
                        \ 1  \
                \end{bmatrix} +
                [1]\begin{bmatrix}
                        \ 1  \ \\
                        \ -1 \ \\
                        \ 1  \
                \end{bmatrix}                    \\
                 & = [\ -1 \ -1 \ 3\ ]^T
        \end{align*}
\end{proof}
\end{document}