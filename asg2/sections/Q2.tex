\documentclass[../main.tex]{subfiles}
\begin{document}

\section{Recall the standard basis of
  $\mathbb{R}^3$, \DS{ \vec{e_1} = [1 \ 0 \ 0]^T,
  \ \vec{e_2} = [0 \ 1 \ 0]^T, \ \vec{e_3} = [0 \ 0 \ 1]^T}}
% -----------------------------------
% Question 2.a
% -----------------------------------
\subsection[Consider the matrix $A$ Does the set of vectors \DS{\{ A\vec{e_1},A\vec{e_2},A\vec{e_3} \}} form a basis for $\mathbb{R}^3$]
{Consider the matrix \DS{
    A = \begin{bmatrix}
      \ 1 & 0 & 2 \ \\
      \ 1 & 3 & 2 \ \\
      \ 0 & 3 & 2 \
    \end{bmatrix}}. Does the set of vectors \DS{S_1 = \{ A\vec{e_1},A\vec{e_2},A\vec{e_3} \}} form a basis for $\mathbb{R}^3$?
}
\fbox{${S_{1}}$ is a basis for $\mathbb{R}^3$}
\begin{proof}
  To see if the vectors in $S_{1}$ are a basis for $\mathbb{R}^3$, we must verify two things \\\\
  (1) $span\{S_{1}\} = \mathbb{R}^3$,
  \begin{equation*}
    [a\ b\ c]^{T} = c_{1}A\vec{e_1} + c_{2}A\vec{e_2} + c_{3}A\vec{e_1}
  \end{equation*}
  $$
    \begin{matrix}
      1c_{1} & + & 0c_{2} & + & 2c_{3} & = & a \ \\
      1c_{1} & + & 3c_{2} & + & 2c_{3} & = & b \ \\
      0c_{1} & + & 3c_{2} & + & 2c_{3} & = & c \
    \end{matrix}
  $$
  (2) $S_{1}$ is linearly independent.
  \begin{equation*}
    [0\ 0\ 0]^{T} = c_{1}A\vec{e_1} + c_{2}A\vec{e_2} + c_{3}A\vec{e_1}
  \end{equation*}
  $$
    \begin{matrix}
      1c_{1} & + & 0c_{2} & + & 2c_{3} & = & 0 \ \\
      1c_{1} & + & 3c_{2} & + & 2c_{3} & = & 0 \ \\
      0c_{1} & + & 3c_{2} & + & 2c_{3} & = & 0 \
    \end{matrix}
  $$
  Thus, we have reduced the problem to showing that the homogenous system (2) has only the trivial solution, and that the nonhomogenous system (1) is consistent for all values c1, c2, c3.
  The augmented matrix has the reduced row echelon form,
  \begin{align*}
    \begin{bmatrix}
      \ 1 & 0 & 0 & \bigm| & 0\ \\
      \ 0 & 1 & 0 & \bigm| & 0\ \\
      \ 0 & 0 & 1 & \bigm| & 0\
    \end{bmatrix} = I_{3}
  \end{align*}
  $\therefore {S_{1}}$ is a basis for $\mathbb{R}^3$
\end{proof}
% -----------------------------------
% Question 2.b
% -----------------------------------
\subsection[Consider the matrix $B$ Does the set of vectors \DS{\{ B\vec{e_1},B\vec{e_2},B\vec{e_3} \}} form a basis for $\mathbb{R}^3$?]
{Consider the matrix \DS{ B = \begin{bmatrix}
      \ 1 & 0 & 2  \ \\
      \ 1 & 3 & 0  \ \\
      \ 0 & 3 & -2 \
    \end{bmatrix}}. Does the set of vectors \DS{S_2 = \{ B\vec{e_1},B\vec{e_2},B\vec{e_3} \}} form a basis for $\mathbb{R}^3$?
}
\fbox{$S_{2}$ is not a basis for $\mathbb{R}^3$}
\begin{proof}
  To see if the vectors in $S_{2}$ are a basis for $\mathbb{R}^3$, we must verify two things \\\\
  (1) $span\{S_{2}\} = \mathbb{R}^3$,
  \begin{equation*}
    [a\ b\ c]^{T} = c_{1}B\vec{e_1} + c_{2}B\vec{e_2} + c_{3}B\vec{e_1}
  \end{equation*}
  $$
    \begin{matrix}
      \  & 1c_{1} & + & 0c_{2} & + & 2c_{3}    & = & a\ \\
      \  & 1c_{1} & + & 3c_{2} & + & 0c_{3}    & = & b\ \\
      \  & 0c_{1} & + & 3c_{2} & + & {-2}c_{3} & = & c\
    \end{matrix}
  $$
  (2) $S_{2}$ is linearly independent.
  \begin{equation*}
    [0\ 0\ 0]^{T} = c_{1}B\vec{e_1} + c_{2}B\vec{e_2} + c_{3}B\vec{e_1}
  \end{equation*}
  $$
    \begin{matrix}
      \  & 1c_{1} & + & 0c_{2} & + & 2c_{3}    & = & 0\ \\
      \  & 1c_{1} & + & 3c_{2} & + & 0c_{3}    & = & 0\ \\
      \  & 0c_{1} & + & 3c_{2} & + & {-2}c_{3} & = & 0\
    \end{matrix}
  $$
  Thus, we have reduced the problem to showing that the homogenous system (2) has only the trivial solution, and that the nonhomogenous system (1) is consistent for all values c1, c2, c3.
  The augmented matrix has the reduced row echelon form,
  \begin{align*}
    \begin{bmatrix}
      \ 1 & 0 & 2            & \bigm| & 0\ \\
      \ 0 & 1 & -\frac{2}{3} & \bigm| & 0\ \\
      \ 0 & 0 & 0            & \bigm| & 0\
    \end{bmatrix}
    \neq I_{3}
  \end{align*}
  $\therefore S_{2}$ is not a basis for $\mathbb{R}^3$
\end{proof}

% -----------------------------------
% Question 2.c
% -----------------------------------
\subsection{Make a conjecture of the form ``\DS{S = \{ A\vec{e_1},A\vec{e_2},A\vec{e_3} \}} forms a basis for $\mathbb{R}^3$ if and only if $A$ \emph{(insert appropriate property of A here)}''.
}
\fbox{
  Conjecture \DS{S = \{ A\vec{e_1},A\vec{e_2},A\vec{e_3} \}} forms a basis for $\mathbb{R}^3$ if and only if $A$ \emph{(is an invertible matrix)}.
}
% -----------------------------------
% Question 2.d (bonus)
% -----------------------------------
\subsection{Bonus: Prove your conjecture.}
\begin{proof}
  \begin{align}
    A       & = AI_{3}                                                 \\
            & = A[\vec{e_1}\ \vec{e_2}\ \vec{e_3}]                     \\
            & = [A\vec{e_1}\ A\vec{e_2}\ A\vec{e_3}]                   \\
    A^{-1}A & = A^{-1}[A\vec{e_1}\ A\vec{e_2}\ A\vec{e_3}]             \\
            & = [A^{-1}A\vec{e_1}\ A^{-1}A\vec{e_2}\ A^{-1}A\vec{e_3}] \\
            & = [I_{3}\vec{e_1}\ I_{3}\vec{e_2}\ I_{3}\vec{e_3}]       \\
            & = [\vec{e_1}\ \vec{e_2}\ \vec{e_3}]                      \\
            & = I_{3}
  \end{align}
\end{proof}
\end{document}

%  \begin{align*}
%    \textrm{Show work.}                      \\
%    A\vec{e_1} & =
%    \begin{bmatrix}
%      \ 1 & 0 & 2 \ \\
%      \ 1 & 3 & 2 \ \\
%      \ 0 & 3 & 2 \
%    \end{bmatrix} \cdot
%    \begin{bmatrix}
%      \ 1 \ \\
%      \ 0 \ \\
%      \ 0 \
%    \end{bmatrix}
%    =
%    (1)\begin{bmatrix}
%      \ 1 \ \\
%      \ 1 \ \\
%      \ 0 \
%    \end{bmatrix}
%    + (0)\begin{bmatrix}
%      \ 0 \ \\
%      \ 3 \ \\
%      \ 3 \
%    \end{bmatrix}
%    + (0)\begin{bmatrix}
%      \ 2 \ \\
%      \ 2 \ \\
%      \ 2 \
%    \end{bmatrix}           \\
%               & = \begin{bmatrix}
%      \ 1 \ \\
%      \ 1 \ \\
%      \ 0 \
%    \end{bmatrix}
%    \implies A\vec{e_2}  = \begin{bmatrix}
%      \ 0 \ \\
%      \ 3 \ \\
%      \ 3 \
%    \end{bmatrix}
%    \implies A\vec{e_3}  = \begin{bmatrix}
%      \ 2 \ \\
%      \ 2 \ \\
%      \ 2 \
%    \end{bmatrix}
%  \end{align*}