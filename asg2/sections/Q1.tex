\documentclass[../main.tex]{subfiles}
\begin{document}

% -----------------------------------
% Question 1.a
% -----------------------------------

\section{Show that \DS{S = \{ 1+2x+x^2, 2 + 9x, 3 + 3x + 4x^2 \} } is a basis for $P_2$ and write \\ 
$p = 2 + 17x - 3x^2$ as a linear combination of vectors in $S$. Finally, write $[p]_S$.}
\subsection{Show that \DS{S = \{ 1+2x+x^2, 2 + 9x, 3 + 3x + 4x^2 \} } is a basis for $P_2$}
\fbox{Solution: vectors $\{\vec{p_{1}}, \vec{p_{2}}, \vec{p_{3}}\}$ are a basis for $P_3$}
\begin{proof}
  The set $S$ = $\{\vec{p_{1}}, \vec{p_{2}}, \vec{p_{3}}\}$ in a vector space $P_2$, is called a basis if. \\\\
  (1) $S$ spans $P_2$.\\
  (2) $S$ is linearly independent.\\\\
  To prove that the vectors span$\{S\}=P_2$ we must show that
  every vector $\vec{p} = a_{0} + a_{1}x + a_{2}x^{2}$ in $P_2$
  can be expressed as $c_{1}\vec{p_{1}} + c_{2}\vec{p_{2}} + c_{3}\vec{p_{3}} = \vec{p}$\\
  \begin{equation}
    c_{1}(1+2x+x^2) + c_{2}(2 + 9x) + c_{3}(3 + 3x + 4x^2) = \vec{v}
  \end{equation}
  $$
    \begin{matrix}
       & 1c_{1} & + & 2c_{2} & + & 3c_{3} & = a_{0} \\
       & 2c_{1} & + & 9c_{2} & + & 3c_{3} & = a_{1} \\
       & 1c_{1} & + & 0c_{2} & + & 4c_{3} & = a_{2}
    \end{matrix}
  $$
  To prove linear independence we must show that $c_{1}\vec{p_{1}} + c_{2}\vec{p_{2}} + c_{3}\vec{p_{3}} = \vec{0}$
  has only the trivial solution.\\
  \begin{equation}
    c_{1}(1+2x+x^2) + c_{2}(2 + 9x) + c_{3}(3 + 3x + 4x^2) = \vec{0}
  \end{equation}
  $$
    \begin{matrix}
       & 1c_{1} & + & 2c_{2} & + & 3c_{3} & = 0 \\
       & 2c_{1} & + & 9c_{2} & + & 3c_{3} & = 0 \\
       & 1c_{1} & + & 0c_{2} & + & 4c_{3} & = 0
    \end{matrix}
  $$
  Thus, we have reduced the problem to showing that the homogenous system (2) has only the trivial solution, and that the nonhomogenous system (1) is consistent for all values c1, c2, c3.
  The two systems have the same coefficient matrix.
  \begin{align*}
    A = \begin{bmatrix}
      \ 1 & 2 & 3 & \bigm| & 0\ \\
      \ 2 & 9 & 3 & \bigm| & 0\ \\
      \ 1 & 0 & 4 & \bigm| & 0\
    \end{bmatrix}
  \end{align*}
  We will prove both results by showing that $\det(A) \neq 0$
  \begin{align*}
    det(A) & = (1)
    \begin{vmatrix}
      \ 2 & 3 \   \\
      \ 9 & 3 \ \
    \end{vmatrix}
    + (4)
    \begin{vmatrix}
      \ 1 & 2 \   \\
      \ 2 & 9 \ \
    \end{vmatrix}                            \\
           & = 1[(2)(3)-(3)(9)] + 4[(1)(9)-(2)(2)] = -1.
  \end{align*}
  This proves that $\{\vec{p_{1}}, \vec{p_{2}}, \vec{p_{3}}\}$ is a basis for $P_2$.
\end{proof}
% -----------------------------------
% Question 1.b
% -----------------------------------
\subsection{Write $\vec{p} = 2 + 17x - 3x^2$ as a linear combination of vectors in $S$}
\fbox{Solution: $\vec{p} = (1)(1+2x+x^2) + (2)(2 + 9x) + (-1)(3 + 3x + 4x^2)$}
\begin{proof}
  The equation $c_{1}\vec{p_{1}} + c_{2}\vec{p_{2}} + c_{3}\vec{p_{3}} = \vec{p}$ which can be written as the linear system
  $$
    \begin{matrix}
       & 1c_{1} & + & 2c_{2} & + & 3c_{3} & = 2  \\
       & 2c_{1} & + & 9c_{2} & + & 3c_{3} & = 17 \\
       & 1c_{1} & + & 0c_{2} & + & 4c_{3} & = -3
    \end{matrix}
  $$
  is an expression for a vector $\vec{p}$ in terms of the basis $S$, with scalars $c1, c2, c3$
  being the coordinates of $\vec{p}$ relative to the basis S. Whose augmented matrix has the reduced row echelon form,
  \begin{align*}
    \begin{bmatrix}
      \ 1 & 0 & 0 & \bigm| & 1  \ \\
      \ 0 & 1 & 0 & \bigm| & 2  \ \\
      \ 0 & 0 & 1 & \bigm| & -1 \
    \end{bmatrix} \\
    c_{1} = 1, c_{2} = 2, c_{3} = -1.
  \end{align*}
  This gives $\vec{p} = (1)(1+2x+x^2) + (2)(2 + 9x) + (-1)(3 + 3x + 4x^2)$
\end{proof}
% -----------------------------------
% Question 1.c
% -----------------------------------
\subsection{Finally, write $[p]_S$.}
\fbox{Solution: $[p]_{s} = [1\ 2\ {-1}]^{T}$}
\begin{proof}
  We use $c_{1}, c_{2}, c_{3}$ from 1.b to construct the coordinate vector $[1\ 2\ {-1}]^{T}$ of $\vec{p}$ relative to $S$.
\end{proof}
\end{document}