\documentclass[12pt]{article}

\usepackage{latexsym}
\usepackage{amsfonts}
\usepackage{amsmath, amsthm, amssymb}

\setlength{\textheight}{10in}
\setlength{\textwidth}{6.5in}
\setlength{\topmargin}{-1.0in}
\setlength{\parskip}{0.15in}
\setlength{\oddsidemargin}{-0.1in}

\newcommand{\DS} [1] {${\displaystyle #1}$}

\author{Michael Walker}
\title{Math 2311 $-$ Assignment 1}
\date{\today}

\begin{document}
\maketitle
\begin{enumerate}
        \item
              Determine if each of the following sets is a vector space.
              \begin{enumerate}
                      \item $V=$\DS{ \left\{ \left[ \begin{array}{c}
                                                    x \\
                                                    y
                                            \end{array} \right] \in\mathbb{R}^2 \ | \ x \geq{y} \right\}}
                            with the usual scalar multiplication
                            and vector addition from $\mathbb{R}^2$
              \end{enumerate}
              \paragraph{Answer:} No, $V$ is not a vector space.
              \begin{proof}
                      let $({x_0}, {y_0}){^T}, ({x_1}, {y_1}){^T} \in V$,
                      and take the vector space operations on $V$
                      to be the usual operations of $vector$
                      addition and $scalar$ multiplication; that is,
                      \begin{align}
                              ({x_0}, {y_0}){^T}+ & ({x_1}, {y_1}){^T} =({x_0}+{x_1}, {y_0}+{y_1}){^T} \\
                                                  & k({x_0}, {y_0}){^T} = ({kx_0}, {ky_0}){^T}
                      \end{align}
                      $V$ is closed under scalar addition since ${x_0 + x_1} \ge {y_0 + y_1}$ \\\\
                      However, by properties of inequalities if the constant,
                      k, is negative, we must reverse the symbol to preserve the inequality relation.

                      Given that k is negative, ${x\geq y} \rightarrow {kx\leq ky}$
              \end{proof}
              \begin{enumerate}
                      \item[(b)] Consider the set $W = \{f \in F(-\infty, \infty) \ | f(1) = 0\}$
                            with the usual scalar multiplication and vector addition from
                            $F(-\infty, \infty)$.
                            Is $W$ a vector space?
              \end{enumerate}
              \paragraph{Answer:} Yes, $W$ is a vector space.
              \begin{proof}
                      Since we know that $F(-\infty, \infty)$ (with the usual operations) is a vector space,
                      and since $W$ is a subset of $F(-\infty, \infty)$ (with the same operations),
                      it suffices to prove that $W$ is a subspace of $F(-\infty, \infty)$.
                      To this end we must show three things:
                      \begin{enumerate}
                              \item That $W$ is non-empty.
                              \item That $W$ is closed under addition.
                              \item That $W$ is closed under scalar multiplication.
                      \end{enumerate}
                      There exists a function $\mathbf{0}$ in $F(-\infty, \infty)$ defined by $\mathbf{0}(x)=0$ for all $x$.
                      Clearly $\mathbf{0}(1)=f(1)=0$ so $W$ is non-empty.

                      Now suppose $f$ and $g$ are two functions in $W$. We must show that $f+g$ is in $W$.
                      \begin{align*}
                              (f+g)(1) & = f(1) + g(1) & \textrm{(definition of addition of functions)} \\
                                       & = 0           & \textrm{($f$ and $g$ are in $W$)}
                      \end{align*}
                      Finally, to show that $W$ is closed under scalar multiplication, suppose $f$ is in $W$ and $k$ is a scalar, then
                      \begin{align*}
                              (kf)(1) & = kf(1) & \textrm{(definition of scalar multiplication on functions)} \\
                                      & = 0     & \textrm{($f$ is in $W$)}
                      \end{align*}
                      so $(kf)$ is in $W$ and $W$ is closed under scalar multiplication.

                      Therefore $W$ is a subspace of $F(-\infty, \infty)$ and hence is a vector space.
              \end{proof}
        \item
              Let $V$ be a vector space.
              \begin{enumerate}
                      \item If $k$ is any scalar, prove that $k\vec{0} = \vec{0}$.
              \end{enumerate}
              \begin{proof}
                      \begin{align*}
                              k(\vec{0} + \vec{u})
                               & = k\vec{0} + k\vec{u}                 & \textrm{(vector space axiom 7)} \\
                              k\vec{u}
                               & = k\vec{0} + k\vec{u}                 & \textrm{(vector space axiom 4)} \\
                              k(\vec{u}) + (-k\vec{u})
                               & = (-k\vec{u}) + (k\vec{0} + k\vec{u}) & \textrm{(vector space axiom 5)} \\
                              \vec{\mathbf{0}}
                               & = (-k\vec{u}) + (k\vec{0} + k\vec{u}) & \textrm{(vector space axiom 5)} \\
                               & = (k\vec{0} + k\vec{u}) + (-k\vec{u}) & \textrm{(vector space axiom 2)} \\
                               & = k\vec{0} + (k\vec{u} + (-k\vec{u})) & \textrm{(vector space axiom 3)} \\
                               & = k\vec{0} + \vec{0}                  & \textrm{(vector space axiom 5)} \\
                               & = \mathbf{k}\vec{\mathbf{0}}          & \textrm{(vector space axiom 4)} \\
                      \end{align*}
              \end{proof}
              \begin{enumerate}
                      \item[(b)] Prove that the zero vector in $V$ is unique.
              \end{enumerate}
              \begin{proof}
                      We must show that there is only one vector, $\vec{0}$, with the property that \\
                      $\vec{0} + \vec{v} = \vec{v} + \vec{0} = \vec{v}$.

                      Suppose $\vec{0_1}$ and $\vec{0}$ are zero vectors in $V$, and $\vec{v}$ is also in $V$. Then
                      \begin{align*}
                              \vec{v}
                               & = \vec{0_1} + \vec{v}                & \textrm{(vector space axiom 4)} \\
                              \vec{v} +(-\vec{v})
                               & = (-\vec{v}) + (\vec{0_1} + \vec{v}) & \textrm{(vector space axiom 5)} \\
                              \vec{\mathbf{0}}
                               & = (-\vec{v}) + (\vec{0_1} + \vec{v}) & \textrm{(vector space axiom 5)} \\
                               & = (\vec{0_1} + \vec{v}) + (-\vec{v}) & \textrm{(vector space axiom 2)} \\
                               & = \vec{0_1} + (\vec{v} + (-\vec{v})) & \textrm{(vector space axiom 3)} \\
                               & = \vec{0_1} + \vec{0}                & \textrm{(vector space axiom 5)} \\
                               & = \vec{\mathbf{0_1}}                 & \textrm{(vector space axiom 4)}
                      \end{align*}
                      Since we have shown that any two zero vectors must be equal to each other,
                      we can conclude that $\vec{0}$ is unique.
              \end{proof}
        \item Determine if each of the following are subspaces of $M_{nn}$
              \begin{enumerate}
                      \item \DS{ \left\{A\in{M_{nn}} \ | det (A) = 0 \right\}}
                            \paragraph{Answer:} No, ${ \left\{A \in M_{nn} \
                                                    | \det (A) = 0 \right\}}$ is not a subspace of $M_{nn}$.
                            \begin{proof}
                                    $\det(A+B) \neq \det (A)+\det (B)$
                                    \begin{equation*}
                                            \det
                                            \begin{vmatrix}
                                                    {1} & {0} \\
                                                    {0} & {0} \\
                                            \end{vmatrix}
                                            {= 0},
                                            \det
                                            \begin{vmatrix}
                                                    {0} & {0} \\
                                                    {0} & {1} \\
                                            \end{vmatrix}
                                            = 0
                                    \end{equation*}
                                    \begin{equation*}
                                            \begin{bmatrix}
                                                    {1} & {0} \\
                                                    {0} & {0} \\
                                            \end{bmatrix}
                                            +
                                            \begin{bmatrix}
                                                    {0} & {0} \\
                                                    {0} & {1} \\
                                            \end{bmatrix}
                                            =
                                            \begin{bmatrix}
                                                    {1} & {0} \\
                                                    {0} & {1} \\
                                            \end{bmatrix}
                                    \end{equation*}
                                    \begin{equation*}
                                            \det
                                            \begin{vmatrix}
                                                    {1} & {0} \\
                                                    {0} & {1} \\
                                            \end{vmatrix}
                                            = 1 \neq 0
                                    \end{equation*}
                            \end{proof}
                      \item \DS{ \left\{A \in{M_{nn}} \ | tr (A) = 0 \right\}}
                            \paragraph{Answer:} Yes, $W={ \left\{A \in M_{nn} \ |
                                    tr(A) = 0 \right\}}$ is a subspace of $M_{nn}$.
                            \begin{proof}
                                    Let [$B = ({b_{ii}})]\in W$ such that $tr(B)=0$ and let ${k}$ be any skalar\\\\
                                    The set $W$ is non empty because, if we let ${a_{ii}=0}$
                                    for all i then $tr(A)=0$ therefore $W$ contains the $\mathbf{0}$ matrices.
                                    It remains to show that W is closed under addition and scalar multiplication
                                    \begin{align*}
                                            tr(A+B) & = \sum_{i = 1}^{n}(a_{ii}+b_{ii})                 \\
                                                    & = \sum_{i = 1}^{n}a_{ii} + \sum_{i = 1}^{n}b_{ii} \\
                                                    & = tr(A) + tr(B)                                   \\
                                                    & = 0 + 0                                           \\
                                                    & = 0.
                                    \end{align*}
                                    \begin{align*}
                                            tr(kA) & = \sum_{i = 1}^{n}(k\cdot a_{ii}) \\
                                                   & = k\cdot \sum_{i = 1}^{n}a_{ii}   \\
                                                   & = k\cdot tr(A)                    \\
                                                   & = k\cdot 0                        \\
                                                   & = 0
                                    \end{align*}
                                    To be clear, if we take some $C = A+B$ such that A and B are in $W$,
                                    then for all $C = (c_{ii})$, the sum will be $0$, so $C$ is also in $W$.
                            \end{proof}
                      \item \DS{ \left\{A \in{M_{nn}} \ | A^T = A \right\}}
                            \paragraph{Answer:} Yes, $W={ \left\{A \in{M_{nn}} \ |
                                    A^T = A \right\}}$ is a subspace of $M_{nn}$.
                            \begin{proof}
                                    Let $[B = (b_{ij})]\in W$ be a square matrix of order $n$ such that $b_{ij} = b_{ji}$, and let ${k}$ be any skalar\\\\
                                    The set $W$ is non empty because, if we let $[A=({a_{ii})]=0}$ for all $i$ then $W$ contains the $\mathbf{0}$ matrices.
                                    It remains to show that $W$ is closed under addition and scalar multiplication\\\\
                                    $C = (A+B)^T = A^T+B^T = A + B$ is in $W$,
                                    and $(k\cdot A)T = k\cdot AT = k\cdot A$ is also in $W$.
                                    Therefore $W$ is non-empty and closed under addition and scalar multiplication.
                            \end{proof}
              \end{enumerate}
\end{enumerate}
\end{document}
